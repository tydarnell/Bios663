\documentclass{article}
\usepackage[utf8]{inputenc}
\usepackage[english]{babel}
\usepackage [autostyle, english = american]{csquotes}
\MakeOuterQuote{"}
\usepackage{graphicx}
\usepackage{enumerate}
\usepackage{float}
\graphicspath{ {} }
\usepackage{mathtools}
\usepackage{amsmath, amsthm, amssymb, amsfonts}
\usepackage{caption}
\usepackage{bm}


% For derivatives
\newcommand{\deriv}[1]{\frac{\mathrm{d}}{\mathrm{d}x} (#1)}

% For partial derivatives
\newcommand{\pderiv}[2]{\frac{\partial #1}{\partial #2}}

% Integral dx
\newcommand{\dx}{\mathrm{d}x}
\newcommand{\cd}{\overset{d}{\to}}
\newcommand{\cp}{\overset{p}{\to}}
\newcommand{\B}{\beta}
\newcommand{\e}{\epsilon}
\newcommand{\limn}{\lim_{n\to \infty}}
\newcommand{\lm}{\lambda}
\newcommand{\sg}{\sigma}
\newcommand{\hb}{\hat{\beta}}
\newcommand{\sumn}{\sum_{i=1}^{n}}
\newcommand{\hth}{\hat{\theta}}
\newcommand{\lra}{\Leftrightarrow}
\newcommand{\prodn}{\prod_{i=1}^{n}}
\newcommand{\dll}[1]{\dfrac{\partial\ell}{\partial{#1}}}
\newcommand{\mle}{\hat{\theta}_{MLE}}
\newcommand{\mm}{\hat{\theta}_{MM}}
\newcommand{\sumx}{\sum_{i=1}^{n}x_i}
\newcommand{\ta}{\theta}
\newcommand{\qe}{ \ ?\ }
\allowdisplaybreaks
\begin{document}
\begin{flushleft}
If X less than full rank $(r<p)$ then collinearity exists among the columns of X\\
$X_{n\times p} = X_{*,(n\times r)}V^{\prime}_{+,(r \times p)}$\\
Suppose we have less than full rank model:\\
$y_{n\times 1}=X_{n\times p}\B_{p\times 1}+\epsilon_{n\times 1}$\\
Define $X_{*,(n\times r)}V_{+,(p\times r)}$
with $\B_{*,(r\times 1)}$\\ Then equivalent full rank model:\\
$y_{n\times 1}=X_{*,(n\times r)}\B_{*,(r\times 1)}+\epsilon_{n\times 1}$ with $\hat{\B_*}=(X^{\prime}_*X_*)^{-1}X_*^{\prime}y$\\
Many possible choices for $V_+$ such as the set of eigenvectors of $X^{\prime}X$ corresponding to non-zero eigenvalues.\\
\end{flushleft}
\end{document}
